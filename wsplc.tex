\documentclass[a4paper,10pt]{article}
\usepackage[margin=0.98in]{geometry}
\usepackage[utf8x]{inputenc}
\usepackage[T1]{fontenc}
%\usepackage[french]{babel}
\usepackage[dvips]{graphicx}
\usepackage{color}
\usepackage{verbatim}
\usepackage{caption}
\usepackage{subcaption}
\usepackage{parskip}

\linespread{1}

\newcommand{\FIXME}[1]{\textcolor{red}{\framebox{FIXME:} #1}}
\newcommand{\TODO}[1]{\textcolor{red}{\framebox{TODO:} #1}}

%\title{We need a project title}
%\author{}
%\date{23 sept. 2016}

\begin{document}
%\maketitle
%\tableofcontents
%\newpage

% Manual title page
\begin{center}
  \huge\textbf{Hybrid smart metering solution}\\
  \vspace{0.5em}
  \small\textsc{G. Bayot Katumba, J. Dubrulle, D. Hauweele, A. Van Laere, Prof. Dr Ir S. Bette}
\end{center}

\section{Project description}

To function optimally, Distribution System Operator~(DSO)
must continually find the balance between production and
consumption of electricity in their network. To this end
smart metering is a solution which relies a smart grid of
sensors to offer a bidirectional communication system
between consumers and producers. The goal of this
communication system is two-fold. First it provides
information on the real-time electricity demand to adapt the
production. Second it offers consumers the means to adapt
their consumption to the current production. In this project
we plan to develop a smart metering application.

Several means of communications are possible among which the
Power Line Communication (PLC). The latter has the advantage
of using an already existing infrastructure which, moreover,
belongs to the DSO, namely the power grid itself. However,
it is not designed to carry high frequency signals, we had
to create standards to adapt to the characteristics of the
network. The G3-PLC is one of this standard.

The G3-PLC is a standard that provides solutions to
characteristics related communication problems from the
power grid especially high attenuation and the existence of
several disruptors which generates noise~\cite{itu_sim2016}. The solution recommended by
the G3-PLC standard is to use all the Smart Meter(SM) as a
relay. Therefore, the standard defined a routing protocol,
the LOADng, which routes data from the various SM through a
mesh network to a Concentrator. However, studies show us
that the operation of this network is dependent on the
density SM~\cite{g3plc_density2015}. The problem
arises especially in a rural environment where the distances
between the various counters are important. What happens if
a meter or a set of meter can not join the mesh network?

One solution would be to use a wireless communication. There
is a plethora of wireless communication system: GPRS, 3G,
LTE, WiMAX, etc. All these solutions are also serious
alternatives to the PLC because they are based on existing
standards and infrastructures and are easy to
deploy. However, they make the DSO dependent on
telecommunications operators. In addition to financial
problems, the strategies of telecom operators, including
trying to stay competitive by always offering the latest
technology to its customers do not always match those of the
DSO. However, with the emergence of open wireless solutions
such as LoRa / LoRaWAN, wireless solutions become again an
option for smart metering applications. The LoRa enables a
reliable and long range (up to 10 km) wireless communication
and low cost. However, this communication has a very modest
rate of a few kilobits per second. For our project, we plan
to use Lora as a backup for the G3-PLC.

The project idea is to create a hybrid modem G3-PLC / LoRa
for smart metering. The concentrator and the smart meter
will have the ability to use both communications. They would
use default G3-PLC, as it provides better throughput (up to
more than 200 kbps). When using PLC is not possible or is
difficult they use LoRa link. Long range communication of
LoRa will compensate the weakness of PLC due to High
attenuation and high noise in power grid. Two scenarios are
possible:

\begin{enumerate}
  \item a single SM, the solution is simple a priori, it would use the LoRa link directly communicate with the concentrator.
  \item or a set of SM isolated and between which there is still a mesh network. The project will have to define the strategy:
    \begin{enumerate}
      \item either repatriate G3-PLC data to a single SM and the latter undertakes to send them to the concentrator via its LoRa link.
      \item or let each smart meter would use his LoRa link to send data directly to the hub.
    \end{enumerate}
\end{enumerate}

In both scenarios, the project will define:

\begin{enumerate}
\item The parameters to be considered to take decision to use LoRa link and the G3-PLC link: PER, LQI, data rate etc.
\item Interactions between upper layers and lower layers to take this decision;
\item Interactions between concentrator and SM, including how the concentrator uses its neighborhood or routing table to make the choice between the G3-PLC and LoRa link.
\item In the second case (set of SM) how to make the choice of the SM that retrieves data to the concentrator.
\end{enumerate}

\section{Team member presentation}

\subsection{Gaston BAYOT KATUMBA}
Gaston is an assistant and a PhD student in the
Electromagnetism and Telecommunications Unit of University
of Mons. He obtained his master’s degree in
telecommunication engineering in 2009 from the Royal
Military Academy (ERM). He is working now on the performance
of routing protocol in PLC communication specially LOADng in
G3-PLC standard.


\subsection{Jeremy DUBRULLE}
​Jeremy Dubrulle is a PhD student at the Computer Networking Lab of the University of Mons (UMons), Belgium, where he also studied Computer Sciences. In the context of his Master thesis, he studied routing in Wireless Sensor Networks, more specifically, Quality of Service and traffic differentiation in 6LoWPAN networks using Contiki OS. His research focuses on routing protocols in WSN.


\subsection{David HAUWEELE}
I am a FRIA research fellow at the Computer Science
Institute of Science Faculty of the University of Mons
(UMONS). I started my PhD in November 2014 under the
supervision of Bruno Quoitin at the Computer Networking Lab
of UMONS and am expected to graduate in 2018.

My main research interests are in the context of the
Internet of Things (IoT) and Wireless Sensor Networks
(WSN). More specifically my research focuses on the
generation of portable, modular and energy-efficient
implementations of network protocol stacks for use in
embedded systems.

In the past, I also worked on the implementation of a Linux
based 6LoWPAN border router using the native
6LoWPAN/802.15.4 subsystems of the Linux kernel. During this
project, the subsystems were still in their early infancies.
I contributed to those subsystems and transceiver drivers to
achieve communication between Linux nodes over a 802.15.4
radio link, and from Linux nodes to embedded nodes
classicaly used in WSN.


\subsection{Aurélien VAN LAERE}
Aurélien is a PhD student working on the effect of
non-Gaussian noises on a G3-PLC communication. He obtained
his Master’s degree in telecommunication engineering in 2011
from the University of Mons. His work enabled him to program
various microcontrollers and to study the performance of
multiple transmission technologies. Lately his work focuses on
the simulation of the G3-PLC PHY layer.


\subsection{Supervisor: Prof. Dr Ir Sébastien BETTE}

\bibliographystyle{abbrv} \bibliography{biblio}

\end{document}
