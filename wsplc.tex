\documentclass[a4paper,10pt]{article}
\usepackage[margin=0.98in]{geometry}
\usepackage[utf8x]{inputenc}
\usepackage[T1]{fontenc}
%\usepackage[french]{babel}
\usepackage[dvips]{graphicx}
\usepackage{color}
\usepackage{verbatim}
\usepackage{caption}
\usepackage{subcaption}
\usepackage{parskip}

\linespread{1}

\newcommand{\FIXME}[1]{\textcolor{red}{\framebox{FIXME:} #1}}
\newcommand{\TODO}[1]{\textcolor{red}{\framebox{TODO:} #1}}

%\title{We need a project title}
%\author{}
%\date{23 sept. 2016}

\begin{document}
%\maketitle
%\tableofcontents
%\newpage

% Manual title page
\begin{center}
  \huge\textbf{Hybrid smart metering solution}\\
  \vspace{0.5em}
  \small\textsc{G. Bayot Katumba, J. Dubrulle, D. Hauweele, A. Van Laere, Prof. Dr Ir S. Bette}
\end{center}

\section{Project description}

To function optimally, Distribution System Operator~(DSO)
must continually find the balance between production and
consumption of electricity in their network. To this end
smart metering is a solution which relies on a smart grid of
sensors offering a bidirectional communication system
between consumers and producers. The goal of this
communication system is two-fold. First it provides
information on the real-time electricity demand to adapt the
production. Second it offers consumers the means to adapt
their consumption to the current production, for example by
starting devices consuming large amounts of electricity
during off-peak hours. In this project we plan to develop a
smart metering application.

A first mean of communication is the Power Line
Communication~(PLC) which uses the already existing grid
infrastructure as communication medium. However this medium
is not suited to carry high frequency signals. Therefore the
G3-PLC\FIXME{ref. needed} standard was developed to adapt
the network protocols to the medium characteristics. This
standard provides solutions to specific communication
problems occuring on power grids, especially the high
attenuation and the existence of several disruptors which
generate noise~\cite{itu_sim2016}. The solution recommended
by the G3-PLC standard is to use all Smart Meter~(SM) as
communication relays. To this end the standard defines a
routing protocol, LOADng, which creates routes from the SM
composing the mesh network to a Concentrator. However,
studies have shown that the operation of this network
depends on the SM density~\cite{g3plc_density2015}. This
problem arises especially in rural environment where the
distances between counters are important. What happens if
the density is too low for SMs to join the mesh network?

Wireless communications can be used to overcome this
limitation. Moreover wireless commmunication systems are
easier to deploy as they do not rely on a pre-existing
infrastructure. There is a plethora of wireless
communication system: GPRS, 3G, LTE, WiMAX, etc. Although
these solutions still constitute serious alternatives to
PLC, they tie the DSO to telecommunications operators. In
addition to a financial cost, the strategies of telecom
operators do not always match those of the DSO. In this
context, the emergence of open wireless solutions such as
LoRa / LoRaWAN offers a viable option for smart metering
applications. Despite its very modest rate of a few kilobits
per second, LoRa enables a reliable long range (up to 10km)
and low cost wireless communication. In this project we plan
to LoRa as a backup communication system to overcome the
limitation of G3-PLC with low density networks.

The project idea is to create a hybrid G3-PLC / LoRa modem
for smart metering. The concentrator and the smart meter
will have the ability to use both communication systems. By
default, they would use G3-PLC, as it provides better
throughput (up to 200 kbps). LoRa would only be used when
PLC is not possible or difficult. Long range communication
of LoRa will compensate the weakness of PLC due to High
attenuation and high noise in power grid. Two scenarios are
possible:

\begin{enumerate}
  \item a single SM, the solution is simple a priori, it would use the LoRa link to directly communicate with the concentrator.
  \item or a set of isolated SM and between which there is still a mesh network. The project will have to define the strategy:
    \begin{enumerate}
      \item either repatriate G3-PLC data to a single SM and the latter undertakes to send them to the concentrator via its LoRa link
      \item or let each smart meter use his LoRa link to send data directly to the hub.
    \end{enumerate}
\end{enumerate}

In both scenarios, the project will define:

\begin{enumerate}
\item The parameters influencing the selection of the medium: PER, LQI, data rate etc.
\item Interactions between upper layers and lower layers to take this decision.
\item Interactions between concentrator and SM, including how the concentrator uses its neighborhood or routing table to make the choice between the G3-PLC and LoRa link.
\item In the second case (set of SM) how to make the choice of the SM that retrieves data to the concentrator.
\end{enumerate}

\section{Team member presentation}

\subsection{Gaston BAYOT KATUMBA}
Gaston is an assistant and a PhD student in the
Electromagnetism and Telecommunications Unit of University
of Mons. He obtained his master’s degree in
telecommunication engineering in 2009 from the Royal
Military Academy (ERM). He is working now on the performance
of routing protocol in PLC communication specially LOADng in
G3-PLC standard.


\subsection{Jeremy DUBRULLE}
​Jeremy Dubrulle is a PhD student at the Computer Networking Lab of the University of Mons (UMons), Belgium, where he also studied Computer Sciences. In the context of his Master thesis, he studied routing in Wireless Sensor Networks, more specifically, Quality of Service and traffic differentiation in 6LoWPAN networks using Contiki OS. His research focuses on routing protocols in WSN.


\subsection{David HAUWEELE}
I am a FRIA research fellow at the Computer Science
Institute of Science Faculty of the University of Mons
(UMONS). I started my PhD in November 2014 under the
supervision of Bruno Quoitin at the Computer Networking Lab
of UMONS and am expected to graduate in 2018.

My main research interests are in the context of the
Internet of Things (IoT) and Wireless Sensor Networks
(WSN). More specifically my research focuses on the
generation of portable, modular and energy-efficient
implementations of network protocol stacks for use in
embedded systems.

In the past, I also worked on the implementation of a Linux
based 6LoWPAN border router using the native
6LoWPAN/802.15.4 subsystems of the Linux kernel. During this
project, the subsystems were still in their early infancies.
I contributed to those subsystems and transceiver drivers to
achieve communication between Linux nodes over a 802.15.4
radio link, and from Linux nodes to embedded nodes
classicaly used in WSN.


\subsection{Aurélien VAN LAERE}
Aurélien is a PhD student working on the effect of
non-Gaussian noises on a G3-PLC communication. He obtained
his Master’s degree in telecommunication engineering in 2011
from the University of Mons. His work enabled him to program
various microcontrollers and to study the performance of
multiple transmission technologies. Lately his work focuses on
the simulation of the G3-PLC PHY layer.


\subsection{Supervisor: Prof. Dr Ir Sébastien BETTE}

\bibliographystyle{abbrv} \bibliography{biblio}

\end{document}
